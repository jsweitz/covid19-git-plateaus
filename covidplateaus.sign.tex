\begin{boxit}
\noindent
\textbf{Significance statement:}\\
In contrast to predictions of conventional epidemic models,
COVID-19 dynamics have asymmetric shapes, with cases and fatalities
declining much more slowly than they rose.
This manuscript evaluates how awareness-driven
behavior modulates epidemic shape.
We find that short-term awareness of fatalities leads
to emergent plateaus, persistent shoulder-like dynamics, and lag-driven
oscillations in a SEIR-like model; consistent with analysis
of US state-level data.
However in contrast to model predictions, 
a joint analysis of fatalities and mobility suggest that
populations relaxed mobility restrictions prior to fatality peaks.
We show that incorporating fatigue and long-term behavior
change dictates post-peak outcomes spanning case resurgence
to sustained epidemic declines.
These findings suggest the need to incorporate
behavior-driven feedback in epidemic models and in public
health campaigns to control COVID-19 spread.
\end{boxit}
