\section{Introduction}
The spread of COVID-19 has elevated the importance of epidemiological
models as a means to forecast both near- and long-term spread. 
In the United States, the Institute for Health Metrics and Evaluation (IHME)
model has emerged as a key influencer of state- and national-level
policy.  The IHME model includes a detailed characterization
of the variation in
hospital bed capacity, ICU beds, and ventilators between and within
states. Predicting the projected strains on underlying
health resources
 is critical to supporting planning efforts (and related
efforts are ongoing at both national~\citep{meyers} and
state levels~\citep{upenn}).  However such projections require
an epidemic `forecast'.  The IHME's epidemic forecast
differs from conventional
epidemic models in a significant way -- IHME assumes
that the cumulative deaths in the COVID-19 epidemic
follow a predefined, Gaussian-like trajectory 
such that deaths go up, reach a peak,
and then go back down.  This trajectory is symmetric
by construction, such that the time it takes to reach the peak 
must equal the time it takes to go back to normal.  For example, the 
IHME model predicts that if the peak is 2 weeks away then in 4 weeks
cases will return to the level of the present, and continue
to diminish rapidly.  But, there is no epidemiological rule
that says epidemics must have one symmetric peak, 
the defunct Farr's Law of Epidemics notwithstanding. 
Farr's law was posited in 1840 before the
germ theory of disease and, like the IHME model, has 
no mechanistic description of the underlying basis for transmission. 
Reserachers revived Farr's Law, in part, to predict the scope of the HIV 
epidemic, e.g., in 1990
the law was used to extrapolate to a peak HIV epidemic size of 200,000 cases by the 
mid-1990s\citep{bregman_1990};
in contrast
there were nearly 2 million new AIDS cases reported last year alone.

Instead, conventional COVID-19 epidemic
models represent populations in terms of their `status' vis
a vis the infectious agent, in this case SARS-CoV-2.  That is, individuals
in a population are represented in different compartments, e.g.,
susceptible, exposed, infectious,
hospitalized, and recovered.  In many cases, epidemic models
might further categorize individuals by age and disease status, or age, occupation,
and disease status, and so on.  As a result, increases in cases
are a result of transmission events between infectious
and susceptible individuals.  The compounding effects 
of new transmission can lead to an exponential increases in cases 
when the basic reproduction number ${\cal{R}}_0>1$ (the
basic reproduction number denotes the average number of new
infections caused by a single, typical individual in an otherwise
susceptible population).  Subsequent
spread, if left unchecked, would yield a single peak -- in theory. That 
peak corresponds to when `herd immunity' is reached, such
that the effective reproduction number, ${\cal{R}}_{eff}=1$.
The effective reprooduction number denotes the number of new
infectious cases caused by a single infectious individual
in a population with pre-existing circulation; this number varies with
time depending on population state, disease characteristics, and 
social behavior.
But, even when herd immunity is reached, there will still be new cases which then
diminish over time, until the epidemic concludes.  
Yet, a single peak paradigm is
only robust insofar as the disease has spread
sufficiently in a population to reach and exceed `herd immunity'.
In this way, the IHME (and other parametric fitting models) make
a dangerous and unsupported assumption: that second peaks
or other long-term trajectories are not possible.  Yet, the converse
is true -- as long as a population remains predominantly immunologically
naive, then the risk of further infection has not passed. 
\begin{figure*}[t!]
\begin{center}
\includegraphics[width=0.3\textwidth]{figseir_baseplat_a1_noname.pdf}
\includegraphics[width=0.3\textwidth]{figseir_baseplat_a2_noname.pdf}
\includegraphics[width=0.3\textwidth]{figseir_baseplat_a4_noname.pdf}
\caption{Infections and deaths per day in a death-awareness based
social distancing model. Awareness varies from $k=1$, $k=2$, to $k=4$ 
in panels (A), (B), and (C).
\label{fig.ID_day}}
\end{center}
\end{figure*}

The Imperial College of London model -- one of the earliest and most influential epidemic models -- is an example of a `conventional' epidemic model
that compares projected epidemic dynamics
between `baseline' scenarios and the outcomes expected
given alternative public health interventions. Despite the many
benefits of public health efforts to reduce transmission: all off their
models suggest a weakness -- early efforts that diminish a peak can be
followed by a second peak or wave of cases precisely because
the disease did not spread early, leaving the majority of the population
susceptible to infection. Yet, buying time for the introduction
of new therapeutics and preserving health care resources suggest
that the benefits of  early intervention steps exceed those of
a `herd immunity' strategy.  Hence, unlike the IHME model,
the ICL model suggests that more than one peak is possible,
precisely because the model integrates latent states of the population
as part of a representation of epidemiological mechanisms.

Yet, even the ICL model has a built-in assumption: that 
behavior changes because of externalities, like lockdowns,
school closings, and so on, that reduce infection.  For a disease
that is already the documented cause of more than 50,000 deaths
in the United States, we posit that individuals, acting
out of a sense of self-preservation, may continue to modify
their behavior.  Hence, here, we use a simple model to
ask the question: what is the anticipated
shape of an epidemic if individuals modify their behavior in direct
response to the impact of a disease at the population level? In doing so,
we build upon earlier work on awareness based models with a
simple assumption: individuals reduce interactions when near-term
death rates are high and increase interactions when near-term death
rates are low.  As a result, we find that epidemic dynamics
can be characterized by long, dangerous plateaus -- rather than peaks.
We suggest that the generic nature of such plateaus should
be considered in developing models that combine
public health interventions that limit interactions with
intrinsic, responses of individuals to the disease state -- and that
can combine to lead to dynamics that defy the near- and long-term
predictions of parametric and even conventional epidemic models.

To begin, consider a SEIR like model
\begin{eqnarray}
\dot{S} &=& -\frac{\beta SI}{\left[1+\left(\dot{D}/D_c\right)^{k}\right]}\\
\dot{E} &=& \frac{\beta SI}{\left[1+\left(\dot{D}/D_c\right)^{k}\right]}-\mu E\\
\dot{I} &=& \mu E-\gamma I \\
\dot{R} &=& (1-f_D)\gamma I\\
\dot{D} &=& f_D\gamma I
\end{eqnarray}
%\begin{eqnarray}
%\dot{S} &=& -\beta SI a(\vec{x}) \\
%\dot{E} &=& \beta SI} a(\vec{x})}-\mu E\\
%\dot{I} &=& \mu E-\gamma I \\
%\dot{R} &=& (1-f_D)\gamma I
%\dot{D} &=& f_D)\gamma I
%\end{eqnarray}
%where $a(\vec{x})$ denotes awareness-based distancing given the
%epidemiological phase space
where $S$, $E$, $I$, $R$, and $D$ denote the proportions of
susecptible, exposed, infectious, recovered, and deaths, respectively.
The awareness-based distancing is controlled by 
the half-saturation constant ($D_c>0$) and
the sharpness of change in the force of infection ($k\geq 1$).
As a result of the proportionalaity
betweeen $\dot{D}$ and $I$, the present model is a variant of a recently proposed
local awareness based distancing model~\citep{eksin_2019}.
Note that the present
model converges to the conventional SEIR model as $D_c\rightarrow \infty$.
\begin{figure*}[t!]
\begin{center}
\includegraphics[width=0.45\textwidth]{figseir_Speak_k1_noname.pdf}
\includegraphics[width=0.435\textwidth]{figseir_Susc_k1_noname.pdf}\\
\includegraphics[width=0.45\textwidth]{figseir_Speak_k2_noname.pdf}
\includegraphics[width=0.435\textwidth]{figseir_Susc_k2_noname.pdf}\\
\includegraphics[width=0.45\textwidth]{figseir_Speak_k4_noname.pdf}
\includegraphics[width=0.435\textwidth]{figseir_Susc_k4_noname.pdf}\\
\caption{Dynamics given variation in the critical fatality awareness
level, $D_c$ and $k$. Panels (A), (C), (E) show deaths/day as a 
function of time. Panels (B), (D), (F) show the fraction of 
susceptibles as a function of time, with respect to a herd immunity
level when only $S=1/{\cal{R}}_0$ remain.
These simulations share the
epidemiological parameters 
$\beta=0.5$ /day, $\mu=1/2$ /day, $\gamma=1/6$/day,
and $f_D=0.01$.
\label{fig.generic}}
\end{center}
\end{figure*}

Typically, epidemics arising in SEIR models have a single case peak, corresponding 
to the point where $\gamma I = \beta S I $ such that 
$S=1/{\cal{R}}_0$, equivalent to when the herd
immunity level proportion of individuals
$1-1/{\cal{R}}_0$ have been infected.
However, when individuals decrease transmission in relationship
to awareness of the aggregate severity of the disease, i.e., $\dot{D}$, 
then the system can `peak' when levels of infected cases are
far from herd immunity, specifically when
\begin{equation}
\gamma I = \frac{\beta SI}{\left[1+\left(\dot{D}/D_c\right)^{k}\right]}.
\end{equation}
In the event that $D_c/\gamma \ll 1$ we anticipate that most of the population
remains susceptible when individual behavior changes markedly
due to awareness of disease severity. Hence, we hypothesize that the
emergence of an
awareness-based peak can occur early, i.e., $S(t)\approx 1$, consistent
with a quasi-stationary equilibrium when
\begin{equation}
\dot{D}^{(q)} \approx D_c\left({\cal{R}}_0-1\right)^{1/k}
\end{equation}
and
\begin{equation}
\dot{I}^{(q)} \approx \frac{D_c}{f_D}\left({\cal{R}}_0-1\right)^{1/k}
\end{equation}
These early onset peak rates should arise not because
of herd immunity but because of changes in behavior. 

We evaluate this hypothesis in
Figure~\ref{fig.ID_day} for $k=1$, $k=2$, and $k=4$
given disease dynamics with $\beta=0.5$ /day, $\mu=1/2$ /day, $\gamma=1/6$
/day,
$f_D=0.01$, and $D_c=10^{-5}$ /day.  In a population of $N=10^7$, the
critical fatality awareness level is equivalent to a rate 100 fatalities
per day, but given the duration of illness, with mean generation
interval of 8 days, then one can consider this critical awareness
level as an approximation of daily fatality rates over week-long periods.
As is evident, the rise and decline from peaks are not symmetric. Instead,
increasing nonlinearity of awareness
$a$ lead to shoulders and, in the limit of $k\gg 1$, plateaus, such that
deaths and cases appear nearly constant and close to the quasi-stationary
equilibrium, declining slowly because of the
depletion of susceptibles.  
We interpret this finding to mean that as the awareness exponent $k$ increases,
individuals become less sensitive to fatality rates
where $\dot{D} < D_c$ and more sensitive to fatality rates where $\dot{D} > D_c$.  
The shoulders and plateaus emerge because relaxation in awareness-based
distancing leads to increases in deaths, and an increase in awareness.

These results suggest a generic outcome: first fatalities will grow
exponential before plataeuing near to the fatality awareness level $D_c$.
In the event that $\gamma D_c$ is sufficiently high then susceptible
depletion will lead to the decline of cases and fatalities.
Figure~\ref{fig.generic} shows
$f_D=0.01$, and $D_c$ values over a range 
equivalent to 5 to 500 deaths/day given a population of $10^7$.
We find that fatalities are sustained at near-constant levels (left)
even as the population remains susceptible at levels far above herd immunity (right).
We observed that as $k$ increases, then fatalities may overshoot
the plateu. This arises because individuals wait to initiate distancing
closer to when a critical fatality rate has been reached.
These overshoots may lead to oscillatory dynamics
when there are larger lags between new cases and fatalities
(whether due to reporting or due to delays arising because of disease
etiology and treatment).

To explore the impacts of oscillations
in a realistic model, we incorporated fatality-based awareness
to the case of COVID-19, including 
detailed information on symptomatic, asymptomatic, hospitalization,
cases, age structure, and age-dependent risk.  In doing so we
consider a model of COVID-19 with a ${\cal{R}}_0=2.32$ and an
incidence fatality rate of 0.82\%. Figure~\ref{fig.covid-plat}}
shows that the insights from the SEIR model generalize to
the case of COVID-19 dynamics, albeit with some differences.  In particular, we
find that deaths exceed $D_c$ because of the lag between 
new infection and fatalities (with a mean of 20 days in the current model).  Further
delays between infection and fatalities can lead to larger oscillations,
as individual behavior changes due to local awareness of deaths.  Reduction in 
contacts when fatality rates exceed the critical awareness level do not translate
into reductions for a period of 20 days, such that deaths can then be driven
below this critical value, then contacts increase, and deaths increase (again with
a lag), and so on.  
\begin{figure}[t!]
\begin{center}
\includegraphics[width=0.45\textwidth]{figbaseline_stir_plat_high_noname.pdf}
\caption{COVID-19 dynamics given awareness-based
social distancing model. Awareness varies from $k=1$, $k=2$, to $k=4$.
Epidemiological and age-specific parameters are from previous work (Weitz et al. shield immunity).
\label{fig.covid-plat}}
\end{center}
\end{figure}

Finally, we note that the scenarios studied, the bulk of the population remains
susceptible throughout the dynamics. Hence, if individuals tire of social
distancing policies, or begin to tolerate higher death rates, then cases can
increase. We also note that long-term awareness would lead to a faster
peak decline (see Eksin et al.). In reality, we expect that individual
behavior is shaped by short- and long-term awareness of risks.  As detailed here,
if short-term awareness predominates, then COVID-19 dynamics may result
in plateau- and shoulder-like behavior, including the possiblity of 
lag-driven oscillations.  We note that these oscillations could be amplified 
in stochastic models.  As a result, passing a `peak' need not imply
the rapid decline of risk. Instead, as cumulative data from outbreak epidemics have already
shown, the asymmetric post-peak dynamics of COVID-19, including
slow declines and plateau-like behavior, may be an emergent
property of awareness-driven epidemiological dynamics.
